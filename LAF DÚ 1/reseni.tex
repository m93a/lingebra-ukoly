%!TEX program = xelatex

\documentclass[]{article}
\usepackage[utf8]{inputenc}
\usepackage{amsmath}
\usepackage{bm}
\usepackage{bbold}
\usepackage{nameref}
\usepackage{hyperref}
\usepackage{listings}
\usepackage[a4paper, total={6in, 9.5in}]{geometry}
\usepackage{fancyhdr}
\usepackage{xfrac}

\makeatletter
\renewcommand*\env@matrix[1][*\c@MaxMatrixCols c]{%
  \hskip -\arraycolsep
  \let\@ifnextchar\new@ifnextchar
  \array{#1}}
\makeatother

\pagestyle{fancy}
\fancyhf{}
\lhead{\leftmark}
\rhead{Michl Grňo}
\cfoot{\thepage}


\renewcommand{\sectionmark}[1]{\markboth{#1}{}}


\renewcommand{\d}[1][]{\ensuremath{\,\mathrm{d}{#1}}}

\newcommand{\dd}[3][]{\ensuremath{\frac{\d^{#1} {#2}}{\d{#3}^{#1}}}}

\newcommand{\pd}[3][]{\ensuremath{\frac{\partial^{#1} {#2}}{\partial {#3}^{#1}}}}

\renewcommand{\U}[1]{\,\mathrm{#1}}

\newcommand{\konst}{\,\mathrm{konst.}}

\newcommand{\Id}[2]{\ensuremath{[\mathrm{Id}]_{#1}^{#2}}}

\newcommand{\e}{\mathrm{e}}




\title{Domácí práce na LAF II.}
\date{6. 5. 2019}
\author{Michal Grňo}

\begin{document}

\maketitle

\begingroup
\addtolength{\jot}{1em}

\section{Cvičení 8 – domácí úloha}

\subsection{Zadání}
Určete $A^n$ pro všechna $n \in \mathbb{N}$ pro matici \(A = \begin{pmatrix}
    2 & 0 & 2 & -2 \\
    1 & 1 & 1 & -1 \\
    -2 & 0 & -3 & 3 \\
    -1 & 0 & -2 & 2
\end{pmatrix} \).

\subsection{Řešení}
Nejprve nalezneme vlastní čísla:
\begin{align*}
    | A - \lambda E |
    &=  \begin{vmatrix}
        2 - \lambda & 0 & 2 & -2 \\
        1 & 1 - \lambda & 1 & -1 \\
        -2 & 0 & -3-\lambda & 3 \\
        -1 & 0 & -2 & 2-\lambda
    \end{vmatrix}
    = (1 - \lambda) (-1)^{2+2} \begin{vmatrix}
        2-\lambda & 2 & -2 \\
        -2 & -3-\lambda & 3 \\
        -1 & -2 & 2-\lambda
    \end{vmatrix}
    = \\
    &= (1-\lambda)\big(\;
        -(2-\lambda)^2 (3+\lambda) \; - \; 8 \; - \; 6 \;
        + \; 6 (2-\lambda) \; + \; 4 (2-\lambda) \; + \; 2 (3+\lambda)
    \;\big) \\
    &= (1-\lambda)( -\lambda^3 + \lambda^2 ) \; = \; \lambda^2 (1-\lambda)^2 \\
    &\implies \lambda_{1,2} = 0, \;\;\; \lambda_{3,4} = 1
\end{align*}

\medskip
\medskip

Následně vyšetříme vlastní podprostor pro $\lambda=0$:
\begin{align*}
    A \sim
    \begin{pmatrix}
        2 & 0 & 2 & -2 \\
        1 & 1 & 1 & -1 \\
        -2 & 0 & -3 & 3 \\
        -1 & 0 & -2 & 2
    \end{pmatrix} \sim
    \begin{pmatrix}
        1 & 0 & 2 & -2 \\
        0 & 1 & -1 & 1 \\
        2 & 0 & 2 & -2 \\
        -2 & 0 & -3 & 3 \\
    \end{pmatrix} \sim
    \begin{pmatrix}
        1 & & & \\
        & 1 & & \\
        & & 1 & -1 \\
        & & & 0 \\
    \end{pmatrix}
    \rightarrow
    \begin{pmatrix}
        0 \\ 0 \\ 1 \\ 1
    \end{pmatrix}
\end{align*}

\medskip
\medskip

Jádro je jednorozměrné, řetízek tedy musí mít délku 2.
\begin{align*}
    \begin{pmatrix}[cccc|c]
        2 & 0 & 2 & -2 & 0\\
        1 & 1 & 1 & -1 & 0\\
        -2 & 0 & -3 & 3 & 1\\
        -1 & 0 & -2 & 2 & 1
    \end{pmatrix} \sim
    \begin{pmatrix}[cccc|c]
        1 & 0 & 2 & -2 & -1\\
        0 & 1 & -1 & -1 & 1\\
        0 & 0 & -2 & 2 & 2\\
        0 & 0 & 1 & -1 & -1
    \end{pmatrix} \sim
    \begin{pmatrix}[cccc|c]
        1 & & & & 1 \\
        & 1 & & & 0 \\
        & & 1 & -1 & -1 \\
        & & & 0 & 0
    \end{pmatrix}
    \rightarrow
    \begin{pmatrix}
        1 \\ 0 \\ -1 \\ 0
    \end{pmatrix}
\end{align*}

\medskip
\medskip

Nyní vyšetříme vlastní podprostor odpovídající $\lambda=1$:
\begin{align*}
    A-1E \sim
    \begin{pmatrix}
        1 & 0 & 2 & -2 \\
        1 & 0 & 1 & -1 \\
        -2 & 0 & -4 & 3 \\
        -1 & 0 & -2 & 1
    \end{pmatrix} \sim
    \begin{pmatrix}
        1 & 0 & 2 & -2 \\
        0 & 0 & -1 & 1 \\
        0 & 0 & 0 & -1 \\
        0 & 0 & 0 & -1
    \end{pmatrix} \sim
    \begin{pmatrix}
        1 & & & \\
        & 0 & & \\
        & & 1 & \\
        & & & 1
    \end{pmatrix}
    \rightarrow
    \begin{pmatrix}
        0 \\ 1 \\ 0 \\ 0
    \end{pmatrix}
\end{align*}

\medskip
\medskip

Jádro je opět jednorozměrné.
\begin{align*}
    \begin{pmatrix}[cccc|c]
        1 & 0 & 2 & -2 & 0\\
        1 & 0 & 1 & -1 & 1\\
        -2 & 0 & -4 & 3 & 0\\
        -1 & 0 & -2 & 1 & 0
    \end{pmatrix} \sim
    \begin{pmatrix}[cccc|c]
        1 & 0 & 2 & -2 & 0\\
        0 & 0 & -1 & 1 & 1\\
        0 & 0 & 0 & -1 & 0\\
        0 & 0 & 0 & -1 & 0
    \end{pmatrix} \sim
    \begin{pmatrix}[cccc|c]
        1 & & & & 2\\
        & 0 & & & 0\\
        & & 1 & & -1\\
        & & & 1 & 0
    \end{pmatrix}
    \rightarrow
    \begin{pmatrix}
        2 \\ 0 \\ -1 \\ 0
    \end{pmatrix}
\end{align*}

\medskip
\medskip

Reprezentace matice v Jordanově bázi $[A]_B^B = J$ a matice přechodu $\Id{B}{K} = V$ jsou tedy:
\begin{align*}
    J =
    \begin{pmatrix}
        0 & 1 \\
          & 0 \\
          &   & 1 & 1 \\
          &   &   & 1
    \end{pmatrix}
    \;\;\;\;\;\;
    V =
    \begin{pmatrix}
        0 & 1 & 0 & 2\\
        0 & 0 & 1 & 0\\
        1 &-1 & 0 & -1\\
        1 & 0 & 0 & 0
    \end{pmatrix}
\end{align*}

\medskip
\medskip

Zbývá nalézt inverzní matici k $V$:
\begin{align*}
    &
    \begin{pmatrix}[cccc|cccc]
        0 & 1 & 0 & 2  &  1 & 0 & 0 & 0 \\
        0 & 0 & 1 & 0  &  0 & 1 & 0 & 0 \\
        1 &-1 & 0 &-1  &  0 & 0 & 1 & 0 \\
        1 & 0 & 0 & 0  &  0 & 0 & 0 & 1
    \end{pmatrix} \sim
    \begin{pmatrix}[cccc|cccc]
        1 & 0 & 0 & 0  &  0 & 0 & 0 & 1 \\
        0 & 1 & 0 & 1  &  0 & 0 &-1 & 1 \\
        0 & 0 & 1 & 0  &  0 & 1 & 0 & 0 \\
        0 & 1 & 0 & 2  &  1 & 0 & 0 & 0 \\
    \end{pmatrix} \sim
    \begin{pmatrix}[cccc|cccc]
        1 & 0 & 0 & 0  &  0 & 0 & 0 & 1 \\
        0 & 1 & 0 & 0  & -1 & 0 &-2 & 2 \\
        0 & 0 & 1 & 0  &  0 & 1 & 0 & 0 \\
        0 & 0 & 0 & 1  &  1 & 0 & 1 &-1 \\
    \end{pmatrix}
\end{align*}
\begin{align*}
    \Id{K}{B} = 
    V^{-1} =
    \begin{pmatrix}
        0 & 0 & 0 & 1 \\
       -1 & 0 &-2 & 2 \\
        0 & 1 & 0 & 0 \\
        1 & 0 & 1 &-1
    \end{pmatrix}
\end{align*}

\medskip
\medskip

Nakonec dopočítáme $A^n$:
\begin{align*}
    A^n
    = (V J V^{-1})^n
    = V J^n V^{-1}
    = V \; \begin{pmatrix}
        0 & 0^{n-1} \\
          & 0 \\
          &   & 1 & n \\
          &   &   & 1
    \end{pmatrix}
    V^{-1}
    \mathrm{,\;kde}\;\;
    0^0 = 1
\end{align*}

\medskip
\medskip

Vydělíme nyní triviální případ $A^1=A$. Pro $n>1$ platí:
\begin{align*}
    A^n =
    \begin{pmatrix}
        0 & 1 & 0 & 2\\
        0 & 0 & 1 & 0\\
        1 &-1 & 0 & -1\\
        1 & 0 & 0 & 0
    \end{pmatrix}
    \begin{pmatrix}
        0 & 0 & 0 & 0 \\
        0 & 0 & 0 & 0 \\
        0 & 0 & 1 & n \\
        0 & 0 & 0 & 1        
    \end{pmatrix}
    \begin{pmatrix}
        0 & 0 & 0 & 1 \\
       -1 & 0 &-2 & 2 \\
        0 & 1 & 0 & 0 \\
        1 & 0 & 1 &-1
    \end{pmatrix}
    =
    \begin{pmatrix}
        2 & 0 & 2 &-2 \\
        n & 1 & n &-n \\
        -1& 0 &-1 & 1 \\
        0 & 0 & 0 & 0
    \end{pmatrix}.
\end{align*}

\subsection{Výsledek}
\begin{align*}
    A^1 &= A, \\
    A^n &= 
    \begin{pmatrix}
        2 & 0 & 2 &-2 \\
        n & 1 & n &-n \\
        -1& 0 &-1 & 1 \\
        0 & 0 & 0 & 0
    \end{pmatrix}, \;\; n \in \{2, 3, 4, ...\}.
\end{align*}


\pagebreak

\section{Cvičení 9 – hodnocená úloha}
\subsection{Zadání}
Nalezněte řešení diferenciální rovnice $\dot{x}(t) = A x(t)$, kde $x(t): \mathbb{R} \rightarrow \mathbb{C}^3$ a
\begin{align*}
    x(0) &=
    \begin{pmatrix}
        1 & -1 & 1
    \end{pmatrix}^\mathrm{T},
    \\
    A &=
    \begin{pmatrix}
        -2 & 1 & 2 \\
        -1 & 0 & 2 \\
        -2 & 0 & 3
    \end{pmatrix}.
\end{align*}

\subsection{Řešení}
Obecným řešením rovnice je $x(t) = \exp(tA) x(0)$, a protože $A = VJV^{-1} \iff \exp(A) = V \exp(J) V^{-1}$, začneme hledáním Jordanova tvaru:
\begin{align*}
    | A - \lambda E |
    = \begin{vmatrix}
        -2-\lambda & 1 & 2 \\
        -1 & -\lambda & 2 \\
        -2 & 0 & 3-\lambda
    \end{vmatrix}
    \;=\; -\lambda^3 + \lambda^2 + \lambda - 1
    \;=\; -(\lambda - 1)^2 (\lambda + 1)
\end{align*}
\begin{align*}
    \lambda_{1,2} = 1, \;\;\;\;\;\; \lambda_3 = -1.
\end{align*}

\medskip
\medskip

Nejprve vyšetříme vlastní prostor pro $\lambda=1$:
\begin{align*}
    A - 1E
    \sim
    \begin{pmatrix}
        -3 & 1 & 2 \\
        -1 &-1 & 2 \\
        -2 & 0 & 2 \\
    \end{pmatrix}
    \sim
    \begin{pmatrix}
        1 & 0 & -1 \\
        0 & 1 & -1
    \end{pmatrix}
    \rightarrow
    \begin{pmatrix}
        2 \\ 2 \\ 2
    \end{pmatrix}
\end{align*}

\medskip
\medskip

Jádro je dimenze 1, to znamená jeden řetízek délky 2.
\begin{align*}
    \begin{pmatrix}[ccc|c]
        -3 & 1 & 2 & 2 \\
        -1 &-1 & 2 & 2 \\
        -2 & 0 & 2 & 2 \\
    \end{pmatrix}
    \sim
    \begin{pmatrix}[ccc|c]
        1 & 0 & -1 & -1 \\
        0 & 1 & -1 & -1
    \end{pmatrix}
    \rightarrow
    \begin{pmatrix}
        -1 \\ -1 \\ 0
    \end{pmatrix}
\end{align*}

\medskip
\medskip

Nyní vyšetříme vlastní prostor odpovídající $\lambda=-1$:
\begin{align*}
    A - (-1)E
    \sim
    \begin{pmatrix}
        -1 & 1 & 2 \\
        -1 & 1 & 2 \\
        -2 & 0 & 4
    \end{pmatrix}
    \sim
    \begin{pmatrix}
        1 & -1 & -2 \\
        -1 & 0 & 2
    \end{pmatrix}
    \sim
    \begin{pmatrix}
        1 & 0 & -2 \\
        0 & 1 & 0
    \end{pmatrix}
    \rightarrow
    \begin{pmatrix}
        2 \\ 0 \\ 1
    \end{pmatrix}
\end{align*}

\medskip
\medskip

Teď nalezneme matici inverzní k přechodové:
\begin{align*}
    \begin{pmatrix}[ccc|ccc]
        2 & -1 & 2  &  1 & 0 & 0 \\
        2 & -1 & 0  &  0 & 1 & 0 \\
        2 &  0 & 1  &  0 & 0 & 1
    \end{pmatrix}
    \sim
    \begin{pmatrix}[ccc|ccc]
        2 & 0 & 1  &  0 & 0 & 1 \\
        0 & 1 & 1  &  0 &-1 & 1 \\
        0 & 0 & 2  &  1 &-1 & 0
    \end{pmatrix}
    \sim
    \begin{pmatrix}[ccc|ccc]
        1 & 0 & 0  &  -\sfrac{1}{4} &  \sfrac{1}{4} & \sfrac{1}{2} \\
        0 & 1 & 0  &  -\sfrac{1}{2} & -\sfrac{1}{2} & 1 \\
        0 & 0 & 1  &  \sfrac{1}{2}  & -\sfrac{1}{2} & 0
    \end{pmatrix}
\end{align*}
\begin{align*}
    A = VJV^{-1} =
    \begin{pmatrix}
        2 & -1 & 2 \\
        2 & -1 & 0 \\
        2 &  0 & 1
    \end{pmatrix}
    \begin{pmatrix}
        1 & 1 &   \\
          & 1 &   \\
          &   & -1
    \end{pmatrix}
    \;\;
    \frac{1}{4}
    \begin{pmatrix}
        -1 & 1 & 2 \\
        -2 & -2 & 4 \\
        2 & -2 & 0
    \end{pmatrix}
\end{align*}

\medskip
\medskip

Nyní vypočteme vektorovou funkci $x(t)$:
\begin{align*}
    x(t) &=
    \exp(tA) v(0) =
    V \exp(tJ) V^{-1} v(0) =
    \\
    &=
    \begin{pmatrix}
        2 & -1 & 2 \\
        2 & -1 & 0 \\
        2 &  0 & 1
    \end{pmatrix}
    \begin{pmatrix}
         \e^t & t\e^t &  \\
              & \e^t  &  \\
              &       & \e^{-t}
    \end{pmatrix}
    \;\;
    \frac{1}{4}
    \begin{pmatrix}
        -1 & 1 & 2 \\
        -2 & -2 & 4 \\
        2 & -2 & 0
    \end{pmatrix}
    \begin{pmatrix}
        1 \\ -1 \\ 1
    \end{pmatrix}
    =
    \\
    &=
    \begin{pmatrix}
        2 & -1 & 2 \\
        2 & -1 & 0 \\
        2 &  0 & 1
    \end{pmatrix}
    \begin{pmatrix}
         \e^t & t\e^t &  \\
              & \e^t  &  \\
              &       & \e^{-t}
    \end{pmatrix}
    \begin{pmatrix}
        0 \\ 1 \\ 1
    \end{pmatrix}
    =
    \begin{pmatrix}
        2t\e^t - \e^t + 2\e^{-t} \\
        2t\e^t - \e^t \\
        2t\e^t + \e^{-t}
    \end{pmatrix}.
\end{align*}

\subsection{Výsledek}
\begin{align*}
    x(t) =
    \begin{pmatrix}
        2t\e^t - \e^t + 2\e^{-t} \\
        2t\e^t - \e^t \\
        2t\e^t + \e^{-t}
    \end{pmatrix}.
\end{align*}



\section{Cvičení 9 – domácí úloha}
\subsection{Zadání}
Určete $\exp(A)$, kde
\begin{align*}
    A = \begin{pmatrix}
        4 & 2 & -5 \\
        6 & 4 & -9 \\
        5 & 3 & -7
    \end{pmatrix}.
\end{align*}

\subsection{Řešení}
Opět začneme nalezením Jordanova tvaru matice:
\begin{align*}
    |A - \lambda E| = \begin{vmatrix}
        4-\lambda & 2 & -5 \\
        6 & 4-\lambda & -9 \\
        5 & 3 & -7-\lambda
    \end{vmatrix} = -\lambda^3 + \lambda^2 = -\lambda^2 (\lambda - 1)
\end{align*}
\begin{align*}
    \lambda_{1,2} = 0, \;\; \lambda_3 = 1
\end{align*}
\begin{align*}
    A \sim
    \begin{pmatrix}
        4 & 2 & -5 \\
        6 & 4 & -9 \\
        5 & 3 & -7
    \end{pmatrix}
    \sim
    \begin{pmatrix}
        4 & 2 & -5 \\
        0 & 2 & -3 \\
        5 & 3 & -7
    \end{pmatrix}
    \sim
    \begin{pmatrix}
        2 & 0 & -1 \\
        0 & 2 & -3 \\
    \end{pmatrix}
    \rightarrow
    \begin{pmatrix}
        1 \\ 3 \\ 2
    \end{pmatrix}
\end{align*}
\begin{align*}
    \begin{pmatrix}[ccc|c]
        4 & 2 & -5 & 1 \\
        6 & 4 & -9 & 3 \\
        5 & 3 & -7 & 2
    \end{pmatrix}
    \sim
    \begin{pmatrix}[ccc|c]
        4 & 2 & -5 & 1 \\
        0 & 2 & -3 & 3 \\
        5 & 3 & -7 & 2
    \end{pmatrix}
    \sim
    \begin{pmatrix}[ccc|c]
        2 & 0 & -1 & -1 \\
        0 & 2 & -3 &  3 \\
    \end{pmatrix}
    \rightarrow
    \begin{pmatrix}
        -1 \\ 0 \\ -1
    \end{pmatrix}
\end{align*}
\begin{align*}
    A - 1E \sim
    \begin{pmatrix}
        3 & 2 & -5 \\
        6 & 3 & -9 \\
        5 & 3 & -8
    \end{pmatrix}
    \sim
    \begin{pmatrix}
        1 & 0 & -1 \\
        5 & 3 & -8 \\
        0 & 1 & -1
    \end{pmatrix}
    \sim
    \begin{pmatrix}
        1 & 0 & -1 \\
        0 & 1 & -1
    \end{pmatrix}
    \rightarrow
    \begin{pmatrix}
        1 \\ 1 \\ 1
    \end{pmatrix}
\end{align*}
\begin{align*}
    \begin{pmatrix}[ccc|ccc]
        1 & -1 & 1  &  1 & 0 & 0 \\
        3 &  0 & 1  &  0 & 1 & 0 \\
        2 & -1 & 1  &  0 & 0 & 1
    \end{pmatrix}
    \sim
    \begin{pmatrix}[ccc|ccc]
        1 & -1 & 1 & 1 & 0 & 0 \\
        0 & 3 & -2 & -3 & 1 & 0 \\
        0 & 1 & -1 & -2 & 0 & 1
    \end{pmatrix}
    \sim
    \begin{pmatrix}[ccc|ccc]
        1 & 0 & 0 & -1 & 0 & 1 \\
        0 & 1 & 0 & 1 & 1 & -2 \\
        0 & 0 & 1 & 3 & 1 & -3
    \end{pmatrix}
\end{align*}
\begin{align*}
    A = V J V^{-1} =
    \begin{pmatrix}
        1 & -1 & 1 \\
        3 &  0 & 1 \\
        2 & -1 & 1
    \end{pmatrix}
    \begin{pmatrix}
        0 & 1 \\
          & 0 \\
          &   & 1 \\
    \end{pmatrix}
    \begin{pmatrix}
        -1 & 0 & 1 \\
         1 & 1 & -2 \\
         3 & 1 & -3
    \end{pmatrix}.
\end{align*}

\medskip
\medskip

Nyní vypočteme exponenciolu matice v Jordanově bázi:
\begin{align*}
    \exp(A) = V \exp(J) V^{-1} = 
    \begin{pmatrix}
        1 & -1 & 1 \\
        3 &  0 & 1 \\
        2 & -1 & 1
    \end{pmatrix}
    \begin{pmatrix}
        1 & 1 \\
          & 1 \\
          &   & \e \\
    \end{pmatrix}
    \begin{pmatrix}
        -1 & 0 & 1 \\
         1 & 1 & -2 \\
         3 & 1 & -3
    \end{pmatrix}
    =
    \begin{pmatrix}
        3\e - 1 & \e & -3\e + 1 \\
        3\e & \e + 3 & -3\e - 3 \\
        3\e - 1 & \e + 1  & -3\e
    \end{pmatrix}.
\end{align*}

\subsection{Výsledek}
\begin{align*}
    \exp(A) =
    \begin{pmatrix}
        3\e - 1 & \e & -3\e + 1 \\
        3\e & \e + 3 & -3\e - 3 \\
        3\e - 1 & \e + 1  & -3\e
    \end{pmatrix}.
\end{align*}




\section{Cvičení 10 – hodnocená úloha}
\subsection{Zadání}
Nechť
\begin{align*}
    u &= \begin{pmatrix}
        1 \\ 1
    \end{pmatrix},
    \\
    v &= \begin{pmatrix}
        1 \\ -2
    \end{pmatrix},
    \\
    w &= \begin{pmatrix}
        -1 \\ 3
    \end{pmatrix},
    \\
    B &= ( \begin{pmatrix}
        1 \\ 2
    \end{pmatrix},
    \begin{pmatrix}
        1 \\ 3
    \end{pmatrix} ),
    \\
    S &= u \otimes v \otimes w,
\end{align*}
vyjádřete $S$ v kanonické bázi $K$ a v bázi $B$ vypočtete její prvek s indexy $2, 1, 2$.


\subsection{Řešení}
Nejprve si vyjádříme matici kontravariantní transformace:
\begin{align*}
    v &= R v'
      = \Id{B}{K} v'
      = \big( \, b_1 \, \big| \, b_2 \, \big) v'
\end{align*}
\begin{align*}
    R &= \begin{pmatrix}
        1 & 1 \\
        2 & 3
    \end{pmatrix}
\end{align*}
\begin{align*}
    \begin{pmatrix}[cc|cc]
        1 & 1 & 1 & 0 \\
        2 & 3 & 0 & 1
    \end{pmatrix}
    \sim
    \begin{pmatrix}[cc|cc]
        1 & 1 & 1 & 0 \\
        0 & 1 & -2 & 1
    \end{pmatrix}
    \sim
    \begin{pmatrix}[cc|cc]
        1 & 0 & 3 & -1 \\
        0 & 1 & -2 & 1
    \end{pmatrix}
\end{align*}
\begin{align*}
    R^{-1} &= \begin{pmatrix}
        3 & -1 \\
        -2 & 1
    \end{pmatrix}
\end{align*}
\begin{align*}
    (R^{-1})^i_j &= 3 \delta^i_1 \delta_j^1 - 2 \delta^i_2 \delta_j^1
            -\delta^i_1 \delta_j^2 + \delta^i_2 \delta_j^2
\end{align*}

\medskip
\medskip

Následně dosadíme do definice $S$:
\begin{align*}
    S^{ijk} &= u^i v^j w^k
    = 1 (\delta_1^j - 2 \delta_2^j) (-\delta_1^k + 3 \delta_2^k)
    = -\delta_1^j \delta_1^k + 2 \delta_2^j \delta_1^k
    + 3 \delta_1^j \delta_2^k - 6 \delta_2^j \delta_2^k
    = \begin{pmatrix}
        -1 &  3 \\
         2 & -6
    \end{pmatrix}^{jk},
\end{align*}

\medskip
\medskip

A provedeme transformaci
\begin{align*}
    (S')^{212} &= (R^{-1})_i^2 \; (R^{-1})_j^1 \; (R^{-1})_k^2 \; S^{ijk} \\
        &= (R^{-1})_i^2 \; (R^{-1})_j^1 \; (5\delta^j_1 - 10\delta^j_2) \\
        &= (R^{-1})_i^2 \; (\delta^i_1 + \delta^i_2) \; 25 \\
        &= (-2 + 1) 25 = -25.
\end{align*}


\subsection{Výsledek}
\begin{align*}
    S^{ijk}
    = -\delta_1^j \delta_1^k + 2 \delta_2^j \delta_1^k
    + 3 \delta_1^j \delta_2^k - 6 \delta_2^j \delta_2^k,
\end{align*}
\begin{align*}
    (S')^{212} = -25.
\end{align*}




\section{Cvičení 10 – domácí úloha}
\subsection{Zadání}
Nalezte duální bázi $B$ k bázi
\begin{align*}
    B^* &= ( b^1,  b^2, b^3 ) \\[-.8em]
    [b^1]_K &= \begin{pmatrix}
        1 & 0 & 2
    \end{pmatrix} \\[-.8em]
    [b^2]_K &= \begin{pmatrix}
        2 & 1 & 1
    \end{pmatrix} \\[-.8em]
    [b^3]_K &= \begin{pmatrix}
        1 & 1 & 0
    \end{pmatrix}.
\end{align*}

\subsection{Řešení}
Duální báze k $B = (b_1, b_2, b_3)$ musí splňovat podmínku:
\begin{align*}
    b^i(b_j) = \delta^i_j.
\end{align*}
\medskip
V kanonické bázi tuto podmínku lze také zapsat maticově:
\begin{align*}
    \begin{pmatrix}[c]
        [b^1]_K \\ \hline
        [b^2]_K \\ \hline
        [b^3]_K
    \end{pmatrix}
    \begin{pmatrix}[c|c|c]
        \; & \; & \;
        \\
        [b_1]_K &
        [b_2]_K &
        [b_3]_K
        \\
        \; & \; & \;
    \end{pmatrix}
    = E.
\end{align*}
\medskip
Úloha je tedy ekvivalentní hledání inverzní matice:
\begin{align*}
    \begin{pmatrix}[ccc|ccc]
        1 & 0 & 2  &  1 & 0 & 0 \\
        2 & 1 & 1  &  0 & 1 & 0 \\
        1 & 1 & 0  &  0 & 0 & 1
    \end{pmatrix}
    \sim
    \begin{pmatrix}[ccc|ccc]
        1 & 0 & 2  &  1 & 0 & 0 \\
        0 & 1 &-1  &  0 &-1 & 2 \\
        0 &-1 & 2  &  1 & 0 &-1
    \end{pmatrix}
    \sim
    \begin{pmatrix}[ccc|ccc]
        1 & 0 & 0  & -1 & 2 &-2 \\
        0 & 1 & 0  &  1 &-2 & 3 \\
        0 & 0 & 1  &  1 &-1 & 1
    \end{pmatrix}
\end{align*}

\subsection{Výsledek}
\begin{align*}
    B = (
        \begin{pmatrix}
            -1 & 2 & -2
        \end{pmatrix}^T
        ,
        \begin{pmatrix}
            1 & -2 & 3
        \end{pmatrix}^T
        ,
        \begin{pmatrix}
            1 & -1 & 1
        \end{pmatrix}^T
    ).
\end{align*}

\endgroup

\end{document}