%!TEX program = xelatex

\documentclass[]{article}
\usepackage[utf8]{inputenc}
\usepackage{amsmath}
\usepackage{bm}
\usepackage{bbold}
\usepackage{nameref}
\usepackage{hyperref}
\usepackage{listings}
\usepackage[a4paper, total={6in, 9.5in}]{geometry}
\usepackage{fancyhdr}
\usepackage{xfrac}
\usepackage{tensor}

\makeatletter
\renewcommand*\env@matrix[1][*\c@MaxMatrixCols c]{%
  \hskip -\arraycolsep
  \let\@ifnextchar\new@ifnextchar
  \array{#1}}
\makeatother

\pagestyle{fancy}
\fancyhf{}
\lhead{\leftmark}
\rhead{Michl Grňo}
\cfoot{\thepage}


\renewcommand{\sectionmark}[1]{\markboth{#1}{}}


\renewcommand{\d}[1][]{\ensuremath{\,\mathrm{d}{#1}}}

\newcommand{\dd}[3][]{\ensuremath{\frac{\d^{#1} {#2}}{\d{#3}^{#1}}}}

\newcommand{\pd}[3][]{\ensuremath{\frac{\partial^{#1} {#2}}{\partial {#3}^{#1}}}}

\renewcommand{\U}[1]{\,\mathrm{#1}}

\newcommand{\konst}{\,\mathrm{konst.}}

\newcommand{\Id}[2]{\ensuremath{[\mathrm{Id}]_{#1}^{#2}}}

\newcommand{\e}{\mathrm{e}}

\newcommand{\kanon}[1]{\ensuremath{\left[#1\right]_\mathrm{K}}}

\newcommand{\mat}[1]{\begin{pmatrix}#1\end{pmatrix}}

\newcommand{\norm}[1]{\left\lVert#1\right\rVert}




\title{Domácí práce na LAF II.}
\date{15. 6. 2019}
\author{Michal Grňo}

\begin{document}

\maketitle

\begingroup
\addtolength{\jot}{1em}


\section{Cvičení 11 – hodnocená úloha}

\subsection{Zadání}
Mějme tenzor $T(A,B,\Gamma) = \Gamma(AB - BA)$, kde $A,B$ jsou reálné antisymetrické 3×3 matice a $\Gamma$ je forma z prostoru duálního. Bázi $K$ zvolíme tak, aby
\begin{align*}
    \begin{pmatrix}
        a \\ b \\ c
    \end{pmatrix}
    =
    \left[
    \;
    \begin{pmatrix}
         0 &  c & b \\
        -c &  0 & a \\
        -b & -a & 0
    \end{pmatrix}
    \;
    \right]_\mathrm{K}
\end{align*}
a skalární součin $g$ je v této bázi $\kanon{g} = \mathrm{E}$.

Vyjádřete tenzor v bázi K: $\kanon{T} = \tensor{T}{_i_j^k}$. Určete obě kontrakce $T_{ij}^i, T_{ij}^j$. Ověřte, zda je tenzor $T_{ijk}$ úplně (anti)symetrický.


\subsection{Řešení}
Označíme si číselné vyjádření vektorů $A,B$ a kovektoru $\Gamma$.
\begin{align*}
    [A]_\mathrm{K} = x^i = \begin{pmatrix}
        a \\ b \\ c
    \end{pmatrix}, \;\;\;
    [B]_\mathrm{K} = y^i = \begin{pmatrix}
        d \\ e \\ f
    \end{pmatrix}, \;\;\;
    [\Gamma]_\mathrm{K^*} = \gamma_i
\end{align*}
\begin{align*}
    [AB - BA]_\mathrm{K} =
    \left[ \;
    \begin{pmatrix}
        0 & ae-bd & cd-af \\
        bd-ae & 0 & bf-ce \\
        af-cd & ce-bf & 0
    \end{pmatrix} \;
    \right]_\mathrm{K}
    =
    \begin{pmatrix}
        bf-ce \\ cd-af \\ ae-bd
    \end{pmatrix}
\end{align*}
\begin{align*}
    T_{ij}^k \; x^i \; y^i \; \gamma_k
    = \gamma_1 \; (x^2 y^3 - x^3 y^2)
    + \gamma_2 \; (x^3 y^1 - x^1 y^3)
    + \gamma_3 \; (x^1 y^2 - x^2 y^1)
    = \gamma (x \times y).
\end{align*}
Odtud vidíme, že spuštěním indexu $k$ získáme smíšený součin:
\begin{gather*}
    T_{ijk} \; x^i \; y^j \; z^k
    = z \cdot (x \times y)
    = \varepsilon_{ijk} \; x^i \; y^j \; z^k
    \\
    T_{ijk} = \varepsilon_{ijk},
\end{gather*}
$T_{ijk}$ je tedy zřejmě úplně antisymetrické.

\bigskip

Vypočteme nyní kontrakce:
\begin{alignat*}{2}
    T_{ij}^i &= g^{ik} \; T_{ijk}
    = g^{ik} \; \varepsilon_{ijk}
    = &\sum_i \varepsilon_{iji}
    = 0
    \\
    T_{ij}^j &=
    &\sum_i \varepsilon_{iij}
    = 0
\end{alignat*}


\subsection{Výsledek}
Tenzor $T$ má v bázi K tvar $\tensor{T}{_i_j^k} = \tensor{\varepsilon}{_i_j^k} = \tensor{g}{^k^l} \; \tensor{\varepsilon}{_i_j_l}$. Z toho důvodu jsou obě kontrakce $T_{ij}^i$ a $T_{ij}^j$ nulové a tenzor $T_{ijk}$ úplně antisymetrický.





\section{Cvičení 11 – domácí úloha}

\subsection{Zadání}
Mějme bázi $B$ prostoru $\mathbb{R}^2$:
\begin{align*}
    B = (e_1, e_2)
    = ( \,
    \begin{pmatrix}
        1 \\ 2
    \end{pmatrix}
    , \;
    \begin{pmatrix}
        1 \\ 3
    \end{pmatrix}
    \, ).
\end{align*}
Nechť $T$ je bilineární forma $T = e^1 \otimes e^2 - e^2  \otimes e^1 + 2e^2 \otimes e^2$ a $g$ skalární součin $\kanon{g} = \begin{pmatrix} 2 & -1 \\ -1 & 1 \end{pmatrix}$. Nalezněte duální bázi $B^*$, matici $\kanon{T}$, vyjádření tenzoru $\widetilde{T} = T(\sharp_g \cdot, \cdot)$ pomocí $\widetilde{T}^i_j \; e_i \otimes e^j$ a jeho kontrakci.

\subsubsection{Řešení}
Pro duální bázi platí podmínka $e^i(e_j) = \delta^i_j$, zapsáno maticově:
\begin{align*}
    \renewcommand{\arraystretch}{1.6}
    \begin{pmatrix}[c]
        [e^1]_K \\ \hline
        [e^2]_K
    \end{pmatrix}
    \renewcommand{\arraystretch}{1}
    \begin{pmatrix}[c|c]
        \; & \;
        \\
        [e_1]_K &
        [e_2]_K
        \\
        \; & \;
    \end{pmatrix}
    = E.
\end{align*}
Duální bázi tedy tvoří řádky inverzní matice: $B^* = \Big((3 \;\; -1),\; (-2 \;\; 1)\Big)$. Dále vypočteme matici
\begin{align*}
    \kanon{T}
    &= \kanon{e^1 \otimes e^2 - e^2  \otimes e^1 + 2e^2 \otimes e^2} \\
    &= \mat{3\\-1} \mat{-2&1} - \mat{-2\\1} \mat{3&-1} + \mat{-2\\1} \mat{-2&1} \\
    &= \mat{8 & -3 \\ -5 & 2}
\end{align*}
Nyní si vyjádříme tenzor $\widetilde{T}$:
\begin{align*}
    \widetilde{T}^i_j &= g^{ik} T_{kj} = \tensor{\Big( \kanon{g^{-1}} \kanon{T} \Big)}{^i_j} \\
    \kanon{\widetilde{T}} &=
    \mat{2 & -1 \\ -1 & 1}^{-1}
    \mat{8 & -3 \\ -5 & 2} =
    \mat{1 & 1 \\ 1 & 2}
    \mat{8 & -3 \\ -5 & 2} =
    \mat{3 & -1 \\ -2 & 1} \\
    \widetilde{T}^i_j &= 3 e_1 \otimes e^1 - e_1 \otimes e^2 -2 e_2 \otimes e^1 + e_2 \otimes e^2
\end{align*}
A jeho kontrakce je
\begin{align*}
    C^1_1 \widetilde{T} = \widetilde{T}^i_i = \mathrm{Tr} \kanon{\widetilde{T}} = 4
\end{align*}





\section{Cvičení 11 – procvičovací úloha 1}

\subsection{Zadání}
Nechť $T$ je tenzor typu $(2,2)$ na $\mathbb{R}^n$ definovaný předpisem
\begin{align*}
    T(u,v,\phi,\psi) = \phi(u) \psi(v).
\end{align*}
Jeho čtyři možné kontrakce zapište ve tvaru $S^i_j \varepsilon_i \otimes \varepsilon^j$, kde $(\varepsilon_i)_1^n$ je kanonická báze $\mathbb{R}^n$.

\subsection{Řešení}
Přepsáním definice do indexové notace získáme
\begin{align*}
    &T(u,v,\phi,\psi) = \phi(u) \psi(v) \\
    &T^{ij}_{kl} \; \phi_i \, \psi_j \, u^k \, v^l = \phi_i u^i \; \psi_j v^j \\
    &T^{ij}_{kl} = \delta^i_k \delta^j_l
\end{align*}
Pro kontrakce tedy platí
\begin{align*}
    &T^{ia}_{ib} = \delta^i_i \delta^a_b = n\delta^a_b \\
    &T^{ai}_{bi} = \delta^a_b \delta^i_i = n\delta^a_b \\
    &T^{ia}_{bi} = \delta^i_b \delta^a_i = \delta^a_b \\
    &T^{ai}_{ib} = \delta^a_i \delta^i_b = \delta^a_b
\end{align*}
Přepsáním do formy $S^i_j \varepsilon_i \otimes \varepsilon^j$ získáme
\begin{align*}
    &C^1_1 T = C^2_2 T = n\delta^a_b \; \varepsilon_a \otimes \varepsilon^b \\
    &C^1_2 T = C^2_1 T = \delta^a_b \; \varepsilon_a \otimes \varepsilon^b
\end{align*}



\section{Cvičení 11 – procvičovací úloha 2}
\subsection{Zadání}
Mějme Minkowského prostor s metrickým tenzorem $\eta = \mathrm{diag}(1,-1,-1,-1)$. Faradayův tenzor $F^\mu_\nu$ má tvar:
\begin{align*}
    F^\mu_\nu = \mat{
        0 & E_x & E_y & E_z \\
        E_x & 0 & B_z & -B_y \\
        E_y & -B_z & 0 & B_x \\
        E_z & B_y & -B_x & 0
    }.
\end{align*}
Vypočtěte $F^{\mu\nu}$ a $F_{\mu\nu}$ a ověřte, že jsou antisymetrické. Vypočtěte $F^{\mu\nu} F_{\mu\nu}$.

\subsection{Řešení}
Zjevně $\eta^{-1} = \eta$. Zdvižení a spuštění indexu vypočteme jako
\begin{align*}
    F^{\mu\nu} &= \tensor{F}{^\mu_\kappa} \eta^{\kappa\nu} = \mat{
        0 & E_x & E_y & E_z \\
        E_x & 0 & B_z & -B_y \\
        E_y & -B_z & 0 & B_x \\
        E_z & B_y & -B_x & 0
    } \mat{1 &&&\\ & -1 &&\\ && -1 &\\ &&& -1}
    = \mat{
        0 & -E_x & -E_y & -E_z \\
        E_x & 0 & -B_z & B_y \\
        E_y & B_z & 0 & -B_x \\
        E_z & -B_y & B_x & 0
    } \\
    F_{\mu\nu} &= \eta_{\mu\kappa} \tensor{F}{^\kappa_\nu}  =
    \mat{1 &&&\\ & -1 &&\\ && -1 &\\ &&& -1}
    \mat{
        0 & E_x & E_y & E_z \\
        E_x & 0 & B_z & -B_y \\
        E_y & -B_z & 0 & B_x \\
        E_z & B_y & -B_x & 0
    }
    = \mat{
        0 & E_x & E_y & E_z \\
        -E_x & 0 & -B_z & B_y \\
        -E_y & B_z & 0 & -B_x \\
        -E_z & -B_y & B_x & 0
    }
\end{align*}
Obě matice jsou zjevně antisymetrické, platí tedy $F^{\mu\nu} = -F^{\nu\mu}$ a $F_{\mu\nu} = -F_{\nu\mu}$. Nakonec vypočteme
\begin{align*}
    F^{\mu\nu} F_{\mu\nu}
    &= -F^{\mu\nu}F_{\nu\mu}
    = -\mathrm{Tr}(
        \mat{
            0 & -E_x & -E_y & -E_z \\
            E_x & 0 & -B_z & B_y \\
            E_y & B_z & 0 & -B_x \\
            E_z & -B_y & B_x & 0
        }
        \mat{
            0 & E_x & E_y & E_z \\
            -E_x & 0 & -B_z & B_y \\
            -E_y & B_z & 0 & -B_x \\
            -E_z & -B_y & B_x & 0
        }
    ) \\
    &= ({E_x}^2 + {E_y}^2 + {E_z}^2) + ({E_x}^2 - {B_z}^2 - {B_y}^2) + ({E_y}^2 - {B_z}^2 - {B_x}^2) + ({E_z}^2 - {B_y}^2 - {B_x}^2) \\
    &= 2 \left( \norm{E}^2 - \norm{B}^2 \right)
\end{align*}





% \section{Cvičení 11 – rozšiřující úloha}

% \subsection{Zadání}
% Dokažte pro $a \in \Lambda^p, b \in \Lambda^q$, že $a \wedge b$ je bilineární, asociativní zobrazení, pro které platí gradovaná komutativita $a \wedge b = (-1)^{p+q} \; b \wedge a$. Dokažte také, že 
% \begin{align*}
%     a = \frac{1}{p!} \, a_{i_1} \, ... \; a_{i_p} \; e^{i_1} \wedge ... \wedge e^{i_p}.
% \end{align*}
% Operátor $\wedge$ je definován jako
% \begin{align*}
%    h 
% \end{align*}

% \subsection{Řešení}



\endgroup

\end{document}