%!TEX program = xelatex

\documentclass[]{article}
\usepackage[utf8]{inputenc}
\usepackage{mathtools}
\usepackage{bm}
\usepackage{bbold}
\usepackage{nameref}
\usepackage{hyperref}
\usepackage{listings}
\usepackage[a4paper, total={6in, 9.5in}]{geometry}
\usepackage{fancyhdr}
\usepackage{xfrac}
\usepackage{tensor}

\makeatletter
\renewcommand*\env@matrix[1][*\c@MaxMatrixCols c]{%
  \hskip -\arraycolsep
  \let\@ifnextchar\new@ifnextchar
  \array{#1}}
\makeatother

\pagestyle{fancy}
\fancyhf{}
\lhead{\leftmark}
\rhead{Michl Grňo}
\cfoot{\thepage}


\renewcommand{\sectionmark}[1]{\markboth{#1}{}}


\renewcommand{\d}[1][]{\ensuremath{\,\mathrm{d}{#1}}}
\newcommand{\dd}[3][]{\ensuremath{\frac{\d^{#1} {#2}}{\d{#3}^{#1}}}}
\newcommand{\pd}[3][]{\ensuremath{\frac{\partial^{#1} {#2}}{\partial {#3}^{#1}}}}

\renewcommand{\U}[1]{\,\mathrm{#1}}

\newcommand{\konst}{\,\mathrm{konst.}}
\newcommand{\e}{\mathrm{e}}

\newcommand{\Ker}{\mathrm{Ker}\;}
\renewcommand{\Im}{\mathrm{Im}\;}
\newcommand{\Tr}{\mathrm{Tr}\;}

\newcommand{\diag}[2]{\mathrm{diag}_{#1}(#2)}
\newcommand{\Id}[2]{\ensuremath{[\mathrm{Id}]_{#1}^{#2}}}

\newcommand{\kanon}[1]{\ensuremath{\left[#1\right]_\mathrm{K}}}
\newcommand{\norm}[1]{\left\lVert#1\right\rVert}
\newcommand{\mat}[1]{\begin{pmatrix}#1\end{pmatrix}}

\newcommand{\recip}[1]{\frac{1}{#1}}
\newcommand{\recipsqrt}[1]{\frac{1}{\sqrt{#1}}}
\newcommand{\x}{\times}
  
  




\title{Domácí práce na LAF II.}
\date{15. 6. 2019}
\author{Michal Grňo}

\begin{document}

\maketitle

\begingroup
\addtolength{\jot}{1em}


\section{Cvičení 5 – hodnocená úloha}
\subsection{Zadání}
Najděte singulární rozklad a pseudoinverzi matice $A = \mat{1&-1\\0&0\\-2&2}$.

\subsection{Řešení}
Nejprve vypočteme matici $A^+A$ a nalezneme její diagonální tvar:
\begin{align*}
  A^+A = \mat{1&0&-2\\-1&0&2}\mat{1&-1\\0&0\\-2&2} = \mat{5&-5\\-5&5},
\end{align*}
\begin{align*}
  |A^+A - \lambda E| = \begin{vmatrix}
      5 - \lambda & -5 \\
      -5 & 5 - \lambda
  \end{vmatrix}
  = \lambda (\lambda - 10),
\end{align*}
\begin{align*}
  \Ker(A^+A) &= k \mat{1 \\ 1}, \;\; k \in \mathbb{R}, \\
  \Ker(A^+A - 10E) &= k \mat{1 \\ -1}, \;\; k \in \mathbb{R}.
\end{align*}
Z toho máme matice $\Sigma$ a $V$:
\begin{align*}
  \Sigma = \mat{\sqrt{10} & 0 \\ 0 & 0 \\ 0 & 0},
  \hspace{2em}
  V = \recipsqrt{2} \mat{1 & 1 \\ -1 & 1}.
\end{align*}
Dále vypočteme první sloupec matice $U$ a zbytek doplníme na ON bázi:
\begin{align*}
  u_1
  \;=\; \recip{\sigma_1} A v_1
  \;=\; \recip{2\sqrt{5}} \mat{1&-1\\0&0\\-2&2} \mat{1\\-1}
  \;=\; \recipsqrt{5} \mat{1\\0\\-2},
\end{align*}
\begin{align*}
  u_2 = \recipsqrt{5} \mat{2\\0\\1},
  \hspace{2em}
  u_3 = \mat{0\\1\\0},
\end{align*}
\begin{align*}
  U = \recipsqrt{5} \mat{1&2&0 \\ 0&0&\sqrt{5} \\ -2&1&0}.
\end{align*}
Singulární rozklad matice $A = U\Sigma V^+$ je tedy:
\begin{align*}
  A =
  \recipsqrt{5} \mat{1&2&0 \\ 0&0&\sqrt{5} \\ -2&1&0}
  \hspace{.5em}
  \mat{\sqrt{10} & 0 \\ 0 & 0 \\ 0 & 0}
  \hspace{.7em}
  \recipsqrt{2} \mat{1 & -1 \\ 1 & 1}.
\end{align*}
Pseudoinverze $A$ je potom
\begin{align*}
  A^\dagger &= V \Sigma^\dagger U^+ \\
  &= \recipsqrt{5} \mat{1&1\\-1&1}
  \hspace{.7em}
  \recipsqrt{10} \mat{1&0&0\\0&0&0}
  \hspace{.7em}
  \recipsqrt{2} \mat{1&0&-2\\2&0&1\\0&\sqrt{5}&0} \\
  &= \recip{10} \mat{1&0&-2\\-1&0&2} \\
  &= \recip{10} \; A^+.
\end{align*}



\section{Cvičení 5 – procvičovací úloha 1}
\subsection{Zadání}
Najděte polární rozklad reálné matice $A=\mat{a&-b\\b&a}$.

\subsection{Řešení}
Nejprve nalezneme singulární rozklad:
\begin{align*}
  A^+A = \mat{a^2 + b^2 & 0 \\ 0 & a^2 + b^2}
\end{align*}
\begin{align*}
  \sigma_{1,2} = \sqrt{a^2 + b^2}
\end{align*}
\begin{align*}
  V = \mat{1&0\\0&1}
  \hspace{2em}
  U = \recipsqrt{a^2 + b^2} \mat{a & -b \\ b & a}
\end{align*}
Následně vyjádříme polární rozklad
\begin{align*}
  A = U \Sigma V^+ = \underbrace{UV^+}_W \; \underbrace{V \Sigma V^+}_P = \underbrace{U}_W \underbrace{\Sigma}_P
\end{align*}
\begin{align*}
  W = \recipsqrt{a^2 + b^2} \mat{a & -b \\ b & a}
  \hspace{2em}
  P = \mat{ \sqrt{a^2 + b^2} & 0 \\ 0 & \sqrt{a^2 + b^2} }
\end{align*}


\section{Cvičení 5 – procvičovací úloha 2}
\subsection{Zadání}
Ukažte, že matice $A = \mat{2 & 3 \\ 0 & 2}$ není diagonalizovatelná a proveďte její singulární a polární rozklad.

\subsection{Řešení}
Snadno demonstrujeme, že $A$ má jedno číslo s algebraickou násobností~2 a geometrickou násobností~1:
\begin{align*}
  |A-\lambda E| =
  \begin{vmatrix}
    2-\lambda & 3 \\ 0 & 2-\lambda
  \end{vmatrix}
  = (\lambda - 2)^2,
\end{align*} 
\begin{align*}
  \Ker(A-\lambda E) =
  \Ker\mat{0&3\\0&0} =
  \langle\mat{1\\0}\rangle.
\end{align*}
Vypočteme tedy singulární rozklad:
\begin{align*}
  A^+A = \mat{4&6\\6&13}
\end{align*}
\begin{align*}
  |A^+A - \lambda E| = (\lambda-16)(\lambda-1)
\end{align*}
\begin{align*}
  \Ker(A^+A-E)
  = \Ker\mat{3&6\\6&12}
  = \Ker\mat{1&2\\0&0}
  = \langle\mat{2\\-1}\rangle
\end{align*}
\begin{align*}
  \Ker(A^+A-16E)
  = \Ker\mat{-12&6\\6&-3}
  = \Ker\mat{2&-1\\0&0}
  = \langle\mat{1\\2}\rangle
\end{align*}
\begin{align*}
  \Sigma = \mat{4&0\\0&1}
  \hspace{2em}
  V = \recipsqrt{5} \mat{1&2\\2&-1}
\end{align*}
\begin{alignat*}{2}
  u_1 &= \recip{4} \mat{2&3\\0&2} \;\; \recipsqrt{5} \mat{1\\2} & \;&=\; \recipsqrt{5} \mat{2\\1} \\
  u_2 &= \recip{1} \mat{2&3\\0&2} \;\; \recipsqrt{5} \mat{2\\-1} & \;&=\; \recipsqrt{5} \mat{1\\-2}
\end{alignat*}
\begin{align*}
  U = \recipsqrt{5} \mat{2&1\\1&-2}
\end{align*}
\begin{align*}
  A = U \Sigma V^+.
\end{align*}
Nakonec vypočteme polární rozklad:
\begin{align*}
  W = UV^+ = \recip{5} \mat{2&1\\1&-2} \mat{1&2\\2&-1} = \recip{5} \mat{4&3\\-3&4}
\end{align*}
\begin{align*}
  P = V \Sigma V^+ = \recip{5} \mat{1&2\\2&-1} \mat{4&0\\0&1} \mat{1&2\\2&-1} = \mat{8&6\\6&17}
\end{align*}

\subsection{Výsledek}
\begin{align*}
  \mat{2&3\\0&2}
  \;\;=\;\; \underbrace{\recip{5}\mat{4&3\\-3&4}}_W \; \underbrace{\recip{5}\mat{8&6\\6&17}}_P
  \;\;=\;\; \underbrace{\recipsqrt{5}\mat{2&1\\1&-2}}_U \; \underbrace{\mat{4&0\\0&1}}_\Sigma \; \underbrace{\recipsqrt{5} \mat{1&2\\2&-1}}_{V^+}.
\end{align*}


\section{Cvičení 5 – procvičovací úloha 3}
\subsection{Zadání}
Mějme matici $X \in \mathbb{R}^{m \x n}$. Jak vypadá pseudoinverze blokové matice $A = \mat{X&0\\0&0}$, resp. $\mat{0&X\\0&0}$, jejíž rozměry jsou $(m+p)\x(n+q)$?

\subsection{Řešení}
Pro rozlišení rozměrů blokových matic budeme používat spodní indexy: například značení $0_{a \x b}$ znamená nulovou matici o $a$ řádcích a $b$ sloupcích. Začneme singulárním rozkladem blokově symetrické varianty $A = \widetilde{U} \widetilde{\Sigma} \widetilde{V}^+$:
\begin{align*}
  A = \mat{
    X_{m \x n} & 0_{m \x q} \\
    0_{p \x n} & 0_{p \x q}
  },
\end{align*}
\begin{align*}
  A^+ A =
  \mat{
    X^+_{n \x m} & 0_{n \x p} \\
    0_{q \x m} & 0_{q \x p}
  }
  \mat{
    X_{m \x n} & 0_{m \x q} \\
    0_{p \x n} & 0_{p \x q}
  }
  =
  \mat{
    (X^+X)_{n \x n} & 0_{n \x q} \\
    0_{q \x n} & 0_{q \x q}
  },
\end{align*}
\begin{align*}
  |A^+A - \lambda E| \; &= \;
  \begin{vmatrix}
    (X^+X - \lambda E)_{n \x n} & 0_{n \x q} \\
    0_{q \x n} & -\lambda E_{q \x q}
  \end{vmatrix} \\
  \; &= \; (-1)^q \;\; \lambda^q \;\; |X^+X - \lambda E|.
\end{align*}
Je zřejmé, že všechna vlastní čísla $X^+X$ budou i vlastními čísly $A^+A$, tato matice bude mít navíc $q$-násobné vlastní číslo 0. Prvních $n$ vlastních vektorů $A^+A$ bude ve tvaru $\widetilde{v_i} = \mat{ (v_i)_{n \x 1} \\ 0_{q \x 1} }$, kde $v_i$ jsou vlastní vektory $X^+X$, zbylé budou ve tvaru $\widetilde{v_i} = \mat{ 0_{n \x 1} \\ (b_i)_{q \x 1}}$, kde $b_i$ je ON báze $\mathbb{C}^q$. Nechť je nyní $X = U \Sigma V^+$ singulární rozklad matice $X$ a $B = \begin{pmatrix}[c|c|c]
  b_1 & ... & b_q
\end{pmatrix}$, potom:
\begin{align*}
  \widetilde{\Sigma}
  = \mat{
    \Sigma_{m \x n} & 0_{m \x q} \\
    0_{p \x n} & 0_{p \x q}
  },
  \hspace{2em}
  \widetilde{V}
  = \mat{
    V_{n \x n} & 0_{n \x q} \\
    0_{q \x n} & B_{q \x q}
  }.
\end{align*}
Dále jde snadno ověřit, že lze zvolit v $\mathbb{C}^{(m+p)}$ takovou ON bázi, aby prvních $m$ sloupců matice $\widetilde{U}$ mělo tvar $\widetilde{u_i} = \mat{(u_i)_{m \x 1} \\ 0_{p \x 1}}$, potom dostáváme
\begin{align*}
  \widetilde{U} = \mat{
    U_{m \x m} & 0_{m \x p} \\
    0_{p \x m} & C_{p \x p}
  },
\end{align*}
kde $C$ je unitární matice.

\medskip

Pseudoinverzní maticí k $A$ tedy bude:
\begin{align*}
  A^\dagger
  &= \widetilde{V}^+ \widetilde{\Sigma}^\dagger \widetilde{U} \\
  &= \mat{
    V^+_{n \x n} & 0_{n \x q} \\
    0_{q \x n} & B^+_{q \x q}
  }
  \mat{
    \Sigma^\dagger_{n \x m} & 0_{n \x p} \\
    0_{q \x m} & 0_{q \x p}
  }
  \mat{
    U_{m \x m} & 0_{m \x p} \\
    0_{p \x m} & C_{p \x p}
  } \\
  &= \mat{
    (V^+ \Sigma^\dagger U)_{n \x m} & 0_{n \x p} \\
    0_{q \x m} & 0_{q \x p}
  } \\
  &= \mat{
    (X^\dagger)_{n \x m} & 0_{n \x p} \\
    0_{q \x m} & 0_{q \x p}
  }.
\end{align*}
V případě, že by $X$ byla čtvercová regulární $n \x n$ matice, platilo by $X^\dagger = X^{-1}$, tedy:
\begin{align*}
  A^\dagger
  = \mat{
    X_{n \x n} & 0_{n \x q} \\
    0_{p \x n} & 0_{p \x q}
  }^\dagger
  = \mat{
    (X^{-1})_{n \x n} & 0_{n \x p} \\
    0_{q \x n} & 0_{q \x p}
  }.
\end{align*}

\vspace{3em}

Nyní vyšetříme blokově nesymetrickou variantu. Nejprve budeme předpokládat, že $p \geq q$, diskusi pro $p < q$ provedeme na konci.
\begin{align*}
  A = \mat{
    0_{m \x q} & X_{m \x n} \\
    0_{p \x q} & 0_{p \x n}
  },
\end{align*}
\begin{align*}
  A^+A = \mat{
    0_{q \x m} & 0_{q \x p} \\
    X^+_{n \x m} & 0_{n \x p}
  }
  \mat{
    0_{m \x q} & X_{m \x n} \\
    0_{p \x q} & 0_{p \x n}
  }
  = \mat{
    0_{q \x q} & 0_{q \x n} \\
    0_{n \x q} & (X^+X)_{n \x n}
  }.
\end{align*}
Nyní si opět vyjádříme singulární rozklad $X = U \Sigma V^+$. Matice $\widetilde{\Sigma}$ a $\widetilde{V}$ najdeme způsobem analogickým k předchozí části, pouze použijeme nestandardní řazení singulárních hodnot v $\widetilde{\Sigma}$:
\footnote{
  Místo standardního tvaru
  $\widetilde{\Sigma} = \diag{(m+p) \x (n+q)}{\sigma_1, \sigma_2, ..., 0, ...}$
  volíme tvar
  $\diag{}{ \underbrace{0, ..., 0}_q, \sigma_1, \sigma_2, ...}$. Díky tomu bude monžné blokově násobit matice $A$ a $V$. Ve vyjádření blokového tvaru $\widetilde{\Sigma}$ bylo využito předpokladu, že $p \geq q$, jinak by byla diagonála "moc dlouhá".
}
\begin{align*}
  \widetilde{\Sigma} = \mat{
    0_{q \x q} & 0_{q \x n} \\
    0_{m \x q} & \Sigma_{m \x n} \\
    0_{(p-q) \x q} & 0_{(p-q) \x n}
  },
  \hspace{2em}
  \widetilde{V}
  = \mat{
    B_{q \x q} & 0_{q \x n} \\
    0_{n \x q} & V_{n \x n}
  }.
\end{align*}
Pro $i \in \{1, ... s\}$, kde $s$ je počet singulárních hodnot matice $X$, platí:
\begin{align*}
  u_i &= \recip{\sigma_i} X v_i,
  \\
  \widetilde{u}_{q+i} &= \recip{\sigma_i} A \widetilde{v}_{q+i} \\
  &= \recip{\sigma_i}
  \mat{
    0_{m \x q} & X_{m \x n} \\
    0_{p \x q} & 0_{p \x n}
  }
  \mat{
    0_{q \x 1} \\ (v_i)_{n \x 1}
  } \\
  &=
  \mat{
    (\recip{\sigma_i} X v_i)_{m \x 1} \\
    0_{p \x 1}
  }
  = \mat{
    (u_i)_{m \x 1} \\
    0_{p \x 1}
  }.
\end{align*}
Protože $(u_1, ..., u_m)$ je ON báze $\mathbb{C}^m$, můžeme použít vzorec $\widetilde{u}_{q+i} = \mat{
  (u_i)_{m \x 1} \\
  0_{p \x 1}
}$ pro hodnoty $i~\in~\{ 1, ..., m \}$, čímž získáme ON bázi $m$-rozměrného podprostoru. Zbývá nám ještě „pokrýt“ $p$-rozměrný podprostor. Zvolíme si tedy libovolnou ON bázi $B = (b_1, ..., b_p)$ v $\mathbb{C}^p$ a její vektory použijeme jako sloupce matic
$
  C =
  \begin{pmatrix}[c|c|c]
    b_1 & ... & b_q
  \end{pmatrix}, \;
  D =
  \begin{pmatrix}[c|c|c]
    b_{q+1} & ... & b_p
  \end{pmatrix}
$.
Matici $\widetilde{U}$ potom lze zapsat ve tvaru:
\begin{align*}
  \widetilde{U} =
  \mat{
    0_{m \x q} & U_{m \x m} & 0_{m \x (p-q)} \\
    C_{p \x q} & 0_{p \x m} & D_{p \x (p-q)}
  }.
\end{align*}
Nyní známe singulární rozklad matice $A$ a můžeme nalézt její pseudoinverzi:
\begin{align*}
  A^\dagger
  &= \widetilde{V} \widetilde{\Sigma}^\dagger \widetilde{U}^+ \\
  &= \mat{
    B_{q \x q} & 0_{q \x n} \\
    0_{n \x q} & V_{n \x n}
  }
  \mat{
    0_{q \x q} & 0_{q \x m} & 0_{q \x (p-q)} \\
    0_{n \x q} & \Sigma^\dagger_{n \x m} & 0_{n \x (p-q)}
  }
  \mat{
    0_{q \x m} &    C^+_{q \x p} \\
    U^+_{m \x m} &    0_{m \x p} \\
    0_{(p-q) \x m} &    D^+_{(p-q) \x p} \\
  } \\
  &= \mat{
    0_{q \x m} & 0_{q \x p} \\
    (V \Sigma U^+)_{n \x m} & 0_{n \x p}
  }
  = \mat{
    0_{q \x m} & 0_{q \x p} \\
    X^\dagger_{n \x m} & 0_{n \x p}
  }.
\end{align*}

V případě, že by bylo $p<q$ by se postupovalo ekvivalentně, pouze $\widetilde{\Sigma}$ a $\widetilde{U}$ by místo tří „blokových řádků“ měly „tři blokové sloupce“. Tento postup vede na stejný výsledek.

\bigskip
\bigskip
\pagebreak

\section{Cvičení 5 – domácí úloha}
\subsection{Zadání}
Určete singulární rozklad, polární rozklad a pseudoinverzi matice $A = \mat{
  1 & -1 & 0 \\
  -1 & 0 & 1 \\
  0 & 1 & -1
}$. Najděte aproximativní řešení soustavy $Ax = b$ s nejmenší normou pro $b = \mat{1 & 1 & 1}^T$.

\subsection{Řešení}
Začneme hledáním singulárního rozkladu:
\begin{align*}
  A^+A = A^2 = \mat{
    2 & -1 & -1 \\
    -1 & 2 & -1 \\
    -1 & -1 & 2
  },
\end{align*}
\begin{align*}
  |A^+A - \lambda E| = -\lambda(\lambda-3)^2,
\end{align*}
\begin{align*}
  \Ker(A^+A) &= \Ker\mat{1&0&-1\\&1&-1} = \langle\mat{1\\1\\1}\rangle, \\
  \Ker(A^+A - 3E) &= \Ker\mat{1&1&1} = \langle\mat{1\\-1\\0},\mat{1\\1\\-2}\rangle,
\end{align*}
\begin{align*}
  \Sigma = \mat{
    \sqrt{3} & 0 & 0 \\
    0 & \sqrt{3} & 0 \\
    0 & 0 & 0
  },
  \hspace{2em}
  V =
  \recipsqrt{6}
  \mat{
    \sqrt{3} & 1 & \sqrt{2} \\
    -\sqrt{3} & 1 & \sqrt{2} \\
    0 & -2 & \sqrt{2}
  },
\end{align*}
\begin{alignat*}{5}
  u_1
  &= \recip{\sigma_1} A v_1 &
  &= \recipsqrt{6} &
  &\mat{
    1 & -1 & 0 \\
    -1 & 0 & 1 \\
    0 & 1 & -1
  } &
  &\mat{1 \\ -1 \\ 0} &
  &= \recipsqrt{6}
  \mat{2\\-1\\-1},
  \\
  u_2
  &= \recip{\sigma_2} A v_2 &
  &= \recip{3\sqrt{6}} &
  &\mat{
    1 & -1 & 0 \\
    -1 & 0 & 1 \\
    0 & 1 & -1
  } &
  &\mat{1 \\ 1 \\ -2} &
  &= \recipsqrt{2}
  \mat{0\\-1\\1},
\end{alignat*}
\begin{align*}
  u_3 = u_2 \x u_1 = \recipsqrt{3} \mat{1\\1\\1},
\end{align*}
\begin{align*}
  U = \recipsqrt{6} \mat{
    2 & 0 & \sqrt{2} \\
    -1 & -\sqrt{3} & \sqrt{2} \\
    -1 & \sqrt{3} & \sqrt{2}
  }.
\end{align*}
\begin{align*}
  A = \underbrace{
    \recipsqrt{6}
    \mat{
      2 & 0 & \sqrt{2} \\
      -1 & -\sqrt{3} & \sqrt{2} \\
      -1 & \sqrt{3} & \sqrt{2}
    }
  }_U
  \;\;\;
  \underbrace{\mat{
    \sqrt{3} & 0 & 0 \\
    0 & \sqrt{3} & 0 \\
    0 & 0 & 0
  }}_\Sigma
  \;\;\;
  \underbrace{
    \recipsqrt{6}
    \mat{
      \sqrt{3} & -\sqrt{3} & 0 \\
      1 & 1 & -2 \\
      \sqrt{2} & \sqrt{2} & \sqrt{2}
    }
  }_{V^+}
  .
\end{align*}
Pokračujeme vyjádřením polárního rozkladu:
\begin{align*}
  W = UV^+ = \recip{6} \mat{
    2 & 0 & \sqrt{2} \\
    -1 & -\sqrt{3} & \sqrt{2} \\
    -1 & \sqrt{3} & \sqrt{2}
  }
  \mat{
    \sqrt{3} & -\sqrt{3} & 0 \\
    1 & 1 & -2 \\
    \sqrt{2} & \sqrt{2} & \sqrt{2}
  }
  =
  \recip{3}
  \mat{
    1+\sqrt{3} & 1-\sqrt{3} & 1 \\
    1-\sqrt{3} & 1 & 1+\sqrt{3} \\
    1 & 1+\sqrt{3} & 1-\sqrt{3}
  }
\end{align*}
\begin{align*}
  P = V \Sigma V^+ =
  \recip{6}
  \mat{
    \sqrt{3} & 1 & \sqrt{2} \\
    -\sqrt{3} & 1 & \sqrt{2} \\
    0 & -2 & \sqrt{2}
  }
  \mat{
    \sqrt{3} & 0 & 0 \\
    0 & \sqrt{3} & 0 \\
    0 & 0 & 0
  }
  \mat{
    \sqrt{3} & -\sqrt{3} & 0 \\
    1 & 1 & -2 \\
    \sqrt{2} & \sqrt{2} & \sqrt{2}
  }
  =
  \recipsqrt{3}
  \mat{
    2 & -1 & -1 \\
    -1 & 2 & -1 \\
    -1 & -1 & 2
  }
\end{align*}
\begin{align*}
  A =
  \underbrace{
    \recip{3}
    \mat{
      1+\sqrt{3} & 1-\sqrt{3} & 1 \\
      1-\sqrt{3} & 1 & 1+\sqrt{3} \\
      1 & 1+\sqrt{3} & 1-\sqrt{3}
    }
  }_W
  \;\;\;
  \underbrace{
    \recipsqrt{3}
    \mat{
      2 & -1 & -1 \\
      -1 & 2 & -1 \\
      -1 & -1 & 2
    }
  }_P
\end{align*}
Dále vypočteme pseudoinverzi:
\begin{align*}
  A^\dagger
  &= V \Sigma^\dagger U^+ \\
  &= \recip{6\sqrt{3}}
  \mat{
    \sqrt{3} & 1 & \sqrt{2} \\
    -\sqrt{3} & 1 & \sqrt{2} \\
    0 & -2 & \sqrt{2}
  }
  \mat{1&0&0\\0&1&0\\0&0&0}
  \mat{
    2 & -1 & -1 \\
    0 & -\sqrt{3} & \sqrt{3} \\
    \sqrt{2} & \sqrt{2} & \sqrt{2}
  } \\
  &=
  \recip{3}
  \mat{
    1 & -1 & 0 \\
    -1 & 0 & 1 \\
    0 & 1 & -1
  }.
\end{align*}
Nakonec vypočteme aproximativní řešení rovnice:
\begin{align*}
  x = A^\dagger b = \recip{3}
  \mat{
    1 & -1 & 0 \\
    -1 & 0 & 1 \\
    0 & 1 & -1
  }
  \mat{1\\1\\1}
  = \mat{0\\0\\0}.
\end{align*}



\section{Cvičení 5 – rozšiřující úloha 2}
\subsection{Zadání}
Mějme matici $A \in \mathbb{R}^{m \x n}$, ukažte, že:
\begin{enumerate}
  \item matice $A A^\dagger$ je matice projekce na $\Im A$,
  \item matice $A^\dagger A$ je matice projekce na $\Im A^T$,
  \item matice $E - A^\dagger A$ je matice projekce na $\Ker A$.
\end{enumerate}

\subsection{Řešení}
Nechť matice $A$ má $s$ singulárních čísel a její singulární rozklad je $A = U \Sigma V^+$, potom $A^\dagger = V \Sigma^\dagger U^+$. Zároveň si zavedeme ON báze $U \subset \mathbb{R}^m, V \subset \mathbb{R}^n$ takové, aby $\Id{V}{\mathrm{K}} = V$ a $\Id{U}{\mathrm{K}} = U$. Kdy budeme mluvit o matici a kdy o bázi, bude zřejmé z kontextu. Nakonec si zavedeme zobrazení $\mathbb{A}: \mathbb{R}^n \rightarrow \mathbb{R}^m$ takové, aby $[\mathbb{A}]_\mathrm{K}^\mathrm{K} = A$, projekci do podprostoru $X$ si označíme $\mathbb{P}_X$ a její matici v kanonické bázi označíme $P_X$.

\medskip

Protože platí
\begin{align*}
  [\mathbb{A}]_\mathrm{K}^\mathrm{K} = A = U \Sigma V^+ = \Id{U}{\mathrm{K}} \Sigma \Id{\mathrm{K}}{V},
\end{align*}
musí být $\Sigma$ vyjádřením $\mathbb{A}$ v bázích $V, U$:
\begin{align*}
  [\mathbb{A}]_V^U = \Sigma.
\end{align*}

Nyní si vyjádříme obraz $\mathbb{A}$:
\begin{align*}
  [\Im\mathbb{A}]^U = \Im \Sigma =
  \left\langle
    \underbrace{
      \mat{1 \\ 0 \\ \vdots},
      \mat{0 \\ 1 \\ \vdots},
      \dots
    }_s
  \right\rangle.
\end{align*}
Matice projekce na $\Im\mathbb{A}$ bude tedy mít v bázi $U$ jednoduchý tvar:
\begin{align*}
  [\mathbb{P}_{\Im\mathbb{A}}]_U^U =
  \mat{
    \underbrace{
      \begin{matrix}
        1 & 0 & \dots \\
        0 & 1 & \dots \\
        \vdots & \vdots
      \end{matrix}
    }_s
    &
    \begin{matrix}
      0 & \dots \\
      0 & \dots \\
      \vdots
    \end{matrix}
  }.
\end{align*}
Vyjádříme-li si teď $A A^\dagger$ pomocí zavedených symbolů, bude vše zřejmé:
\begin{align*}
  A A^\dagger
  &= (U \Sigma V^+)(V \Sigma^\dagger U^+) = U \Sigma \Sigma^\dagger U^+ \\
  &= U \mat{
    \underbrace{
      \begin{matrix}
        1 & 0 & \dots \\
        0 & 1 & \dots \\
        \vdots & \vdots
      \end{matrix}
    }_s
    &
    \begin{matrix}
      0 & \dots \\
      0 & \dots \\
      \vdots
    \end{matrix}
  } U^+ \\
  &= \Id{U}{\mathrm{K}} [\mathbb{P}_{\Im\mathbb{A}}]_U^U \Id{\mathrm{K}}{U}
  = [\mathbb{P}_{\Im\mathbb{A}}]_K^K = P_{\Im A}.
\end{align*}

Stejným způsobem, jakým jsme si zavedli zobrazení $\mathbb{A}: \mathbb{R}^n \rightarrow \mathbb{R}^m$, můžeme zavést i zobrazení $\mathbb{A}^T: \mathbb{R}^m \rightarrow \mathbb{R}^n$.
\begin{align*}
  [\mathbb{A}^T]_\mathrm{K}^\mathrm{K}
  = A^T = (U \Sigma V^+)^T
  = V \Sigma^T U^+,
\end{align*}
\begin{align*}
  [\Im \mathbb{A}^T]^V = \Im \Sigma^T = \left\langle
  \underbrace{
    \mat{1 \\ 0 \\ \vdots},
    \mat{0 \\ 1 \\ \vdots},
    \dots
  }_s
\right\rangle,
\end{align*}
\begin{align*}
  [\mathbb{P}_{\Im\mathbb{A}^T}]_V^V =
  \mat{
    \underbrace{
      \begin{matrix}
        1 & 0 & \dots \\
        0 & 1 & \dots \\
        \vdots & \vdots
      \end{matrix}
    }_s
    &
    \begin{matrix}
      0 & \dots \\
      0 & \dots \\
      \vdots
    \end{matrix}
  }.
\end{align*}
\begin{align*}
  A^\dagger A
  &= (V \Sigma^\dagger U^+)(U \Sigma V^+)
  = V \Sigma^\dagger \Sigma V^+
  = V \Sigma^T (\Sigma^\dagger)^T V^+ \\
  &= \Id{V}{\mathrm{K}} [\mathbb{P}_{\Im\mathbb{A}^T}]_V^V \Id{\mathrm{K}}{V}
  = [\mathbb{P}_{\Im\mathbb{A}^T}]_K^K
  = P_{\Im A^T}.
\end{align*}

\medskip

Poslední tvrzení triviálně vyplývá z
\begin{align*}
  \Ker A = (\Im A^T)^\perp,
\end{align*}
\begin{align*}
  P_{X^\perp} = E - P_X.
\end{align*}
\begin{align*}
  P_{\Ker A} = P_{(\Im A^T)^\perp} = E - P_{\Im A^T} = E - A^\dagger A.
\end{align*}



\endgroup

\end{document}